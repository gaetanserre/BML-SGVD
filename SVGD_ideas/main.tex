\documentclass[runningheads,a4paper]{llncs}

\setcounter{tocdepth}{3}
\usepackage{natbib}
\usepackage[english]{babel}
\usepackage{graphicx}
\usepackage{subcaption}
\usepackage{algorithm}
\usepackage{algpseudocode}
\usepackage{amsmath, amssymb, mathtools, amsthm}
\usepackage{hyperref}
\usepackage{url}
\usepackage[pages=some]{background}
\usepackage[margin=2.8cm]{geometry}
\usepackage{lipsum}
\graphicspath{{figures/}}
\definecolor{MVA}{HTML}{652475}

\newcommand\BackGroundImage[1]{%
\BgThispage
\backgroundsetup{
    scale=1,angle=0,position={current page.north}, hshift=-7.5cm, 
    vshift=-1.5cm, opacity=1,
    contents={%
    \includegraphics[width=5cm]{#1}
}}}

\newcommand{\R}{\mathbb{R}}
\newcommand{\E}{\mathbb{E}}
\newcommand{\X}{\mathcal{X}}
\newcommand{\A}{\mathcal{A}}
\newcommand{\Hf}{\mathcal{H}}
\newcommand{\diff}[2]{\frac{\partial #1}{\partial #2}}
\DeclarePairedDelimiter{\norm}{\|}{\|}
\DeclarePairedDelimiter{\pare}{(}{)}
\DeclarePairedDelimiter{\bra}{\{}{\}}
\DeclarePairedDelimiter{\sq}{[}{]}
\renewcommand\qedsymbol{$\blacksquare$}

\begin{document} 

\mainmatter
\titlerunning{Stein Variational Gradient Descent main ideas}
\title{Stein Variational Gradient Descent main ideas}

\author{Gaëtan Serré}
\authorrunning{Serré}

\institute{École Normale Supérieure Paris-Saclay\\
Master Mathématiques, Vision, Apprentissage\\
gaetan.serre@ens-paris-saclay.fr}

\toctitle{Stein Variational Gradient Descent main ideas}
\tocauthor{Gaëtan Serré}
\maketitle

\BackGroundImage{MVA.pdf}

\section{Goal}
Given a smooth density $\pi$ supported on $\X \subseteq \R^d$,
find $\mu$ on $\X$ as close as possible to $\pi$.

\section{Stein framework}
\paragraph{\bf Stein identity:}
Let $\A_\pi$ a Stein operator s.t.
$$
\A_\pi \phi = \nabla \log \pi(\cdot)^\top \phi(\cdot) + \nabla \cdot \phi(\cdot)
$$
with $\phi(x) = [\phi_1(x), ..., \phi_d(x)]^\top$.
Then, if $\phi$ is in the Stein class f $\pi$ i.e. $\phi(x)\pi(x) = 0$ for all $x \in \partial \X$
if $\X$ is compact or $\lim_{||x|| \to \infty} \phi(x) \pi(x) = 0$ if $\X = \R^d$, we have:
\begin{equation}
  \E_{x \sim \pi}[\A_\pi \phi (x)] = 0
  \label{eq:stein_id}
\end{equation}
\begin{proof}
  \begin{align*}
    \E_{x \sim \pi}[\A_\pi \phi (x)] &=
      \int_\X \pare*{ \nabla \log \pi(\cdot)^\top \phi(\cdot) + \nabla \cdot \phi(\cdot) } \pi(x) dx \\
    &= \int_\X \nabla \log \pi(\cdot)^\top \phi(\cdot) \pi(x) dx + \int_\X \nabla \cdot \phi(\cdot) \pi(x) dx \\
    &= \int_\X \nabla \log \pi(\cdot)^\top \phi(\cdot) \pi(x) dx +
      \int_\X \sum_{k=1}^d \diff{\phi_k}{x_k} \pi(x) dx \\
    &= \int_\X \nabla \log \pi(\cdot)^\top \phi(\cdot) \pi(x) dx +
    \sum_{k=1}^d \pare*{ \sq*{\pi(x) \phi_k(x)}_\X - \int_\X \diff{\pi(x)}{x_k} \phi_k(x) dx}\\
    &= \int_\X \nabla \log \pi(\cdot)^\top \phi(\cdot) \pi(x) dx -
      \int_\X \sum_{k=1}^d \diff{\pi(x)}{x_k} \phi_k(x) dx\\
    &= \int_\X \sum_{k=1}^d \diff{\pi(x)}{x_k} \phi_k(x) -
    \sum_{k=1}^d \diff{\pi(x)}{x_k} \phi_k(x) dx\\
    &= 0
  \end{align*}
\end{proof}

Now, let $\mu$ a smooth density supported on $\X$ different from $\pi$. Now, Eq.~\ref{eq:stein_id} do not hold
anymore with $x \sim \mu$. However, we can use $\E_{x \sim \mu}[\A_\pi \phi (x)]$ as a discrepancy
measure between $\mu$ and $\pi$, as its magnitude relates to how different $\mu$ and $\pi$ are
[\cite{https://doi.org/10.48550/arxiv.1608.04471} \& \cite{https://doi.org/10.48550/arxiv.1704.07520}].
The objective becomes:
\begin{equation}
  \mu^* = \arg\min_{\mu} \: \mathbb{S}(\mu, \pi) =
    \arg\min_{\mu} \: \max_{\phi \in \Hf} \bra*{ \E_{x \sim \mu}[\A_\pi \phi (x)] }  
  \label{eq:stein_obj}
\end{equation}
As $\mathbb{S}(\mu, \pi) = 0$ iff $\mu = \pi$ and $\mathbb{S}(\mu, \pi) > 0$ otherwise with $\Hf$ sufficiently large.
The choice of $\Hf$ is therefore crucial. One way to ensure $\Hf$ is both rich enough and computationally tractable
is to let $\Hf$ be a RKHS.


\bibliography{refs}
\bibliographystyle{plainnat}



\end{document} 
