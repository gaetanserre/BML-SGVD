\documentclass[runningheads,a4paper]{llncs}

\setcounter{tocdepth}{3}
\usepackage{natbib}
\usepackage[english]{babel}
\usepackage{graphicx}
\usepackage{subcaption}
\usepackage{algorithm}
\usepackage{algpseudocode}
\usepackage{amsmath, amssymb, mathtools, amsthm}
\usepackage{hyperref}
\usepackage{url}
\usepackage[pages=some]{background}
\usepackage[margin=2.8cm]{geometry}
\usepackage{lipsum}
\graphicspath{{figures/}}
\definecolor{MVA}{HTML}{652475}

\newcommand\BackGroundImage[1]{%
\BgThispage
\backgroundsetup{
    scale=1,angle=0,position={current page.north}, hshift=-7.5cm, 
    vshift=-1.5cm, opacity=1,
    contents={%
    \includegraphics[width=5cm]{#1}
}}}

\newcommand{\R}{\mathbb{R}}
\newcommand{\E}{\mathbb{E}}
\newcommand{\Sr}{\mathbb{S}}
\newcommand{\X}{\mathcal{X}}
\newcommand{\A}{\mathcal{A}}
\newcommand{\Hr}{\mathcal{H}}
\newcommand{\diff}[2]{\frac{\partial #1}{\partial #2}}
\DeclarePairedDelimiter{\norm}{\|}{\|}
\DeclarePairedDelimiter{\pare}{(}{)}
\DeclarePairedDelimiter{\bra}{\{}{\}}
\DeclarePairedDelimiter{\sq}{[}{]}
\renewcommand\qedsymbol{$\blacksquare$}

\begin{document} 

\mainmatter
\titlerunning{Stein Variational Gradient Descent main ideas}
\title{Stein Variational Gradient Descent main ideas}

\author{Gaëtan Serré}
\authorrunning{Serré}

\institute{École Normale Supérieure Paris-Saclay\\
Master Mathématiques, Vision, Apprentissage\\
gaetan.serre@ens-paris-saclay.fr}

\toctitle{Stein Variational Gradient Descent main ideas}
\tocauthor{Gaëtan Serré}
\maketitle

\BackGroundImage{MVA.pdf}

\section{Goal}
Given a smooth density $\pi$ supported on $\X \subseteq \R^d$,
find $\mu$ on $\X$ as close as possible to $\pi$.

\section{Stein framework}
\paragraph{\bf Stein identity:}
Let $\A_\pi$ a Stein operator s.t.
$$
\A_\pi \phi = \nabla \log \pi(\cdot)^\top \phi(\cdot) + \nabla \cdot \phi(\cdot)
$$
with $\phi(x) = [\phi_1(x), ..., \phi_d(x)]^\top$.
Then, if $\phi$ is in the Stein class f $\pi$ i.e. $\phi(x)\pi(x) = 0$ for all $x \in \partial \X$
if $\X$ is compact or $\lim_{||x|| \to \infty} \phi(x) \pi(x) = 0$ if $\X = \R^d$, we have:
\begin{equation}
  \E_{x \sim \pi}[\A_\pi \phi (x)] = 0
  \label{eq:stein_id}
\end{equation}
\begin{proof}
  \begin{align*}
    \E_{x \sim \pi}[\A_\pi \phi (x)] &=
      \int_\X \pare*{ \nabla \log \pi(\cdot)^\top \phi(\cdot) + \nabla \cdot \phi(\cdot) } \pi(x) dx \\
    &= \int_\X \nabla \log \pi(\cdot)^\top \phi(\cdot) \pi(x) dx + \int_\X \nabla \cdot \phi(\cdot) \pi(x) dx \\
    &= \int_\X \nabla \log \pi(\cdot)^\top \phi(\cdot) \pi(x) dx +
      \int_\X \sum_{k=1}^d \diff{\phi_k}{x_k} \pi(x) dx \\
    &= \int_\X \nabla \log \pi(\cdot)^\top \phi(\cdot) \pi(x) dx +
    \sum_{k=1}^d \pare*{ \sq*{\pi(x) \phi_k(x)}_\X - \int_\X \diff{\pi(x)}{x_k} \phi_k(x) dx} \\
    &= \int_\X \nabla \log \pi(\cdot)^\top \phi(\cdot) \pi(x) dx -
      \int_\X \sum_{k=1}^d \diff{\pi(x)}{x_k} \phi_k(x) dx \\
    &= \int_\X  \pi(x) \sum_{k=1}^d \diff{\log \pi(x)}{x_k}\phi_k(x) -
    \pi(x) \sum_{k=1}^d \diff{\log \pi(x)}{x_k} \phi_k(x) dx \;\; \text{(log trick)} \\
    &= 0
  \end{align*}
\end{proof}

Now, let $\mu$ a smooth density supported on $\X$ different from $\pi$. Now, Eq.~\ref{eq:stein_id} do not hold
anymore with $x \sim \mu$. However, we can use $\E_{x \sim \mu}[\A_\pi \phi (x)]$ as a discrepancy
measure between $\mu$ and $\pi$, as its magnitude relates to how different $\mu$ and $\pi$ are
(see \cite{https://doi.org/10.48550/arxiv.1608.04471} \& \cite{https://doi.org/10.48550/arxiv.1704.07520}).
Indeed, if we assume $\phi$ to be in the Stein class of $\mu$ as well (this is mild condition as
$\pi$ and $\mu$ are two densities on $\X$, one can choose $\phi$ to be in the Stein class
of all distribution on $\X$. E.g. if $\X = \R^d$, one can pick $\phi(x) = \exp\sq*{-\| x - y \|^2}$),
we have:
\begin{equation}
  \begin{split}
    \E_{x \sim \mu}[\A_\pi \phi (x)] &=
      \int_\X \pare*{ \nabla \log \pi(x)^\top \phi(x) + \nabla \cdot \phi(x) } \mu(x) dx \\
    &= \int_\X \nabla \log \pi(x)^\top \phi(x) \mu(x) dx +
    \sum_{k=1}^d \pare*{ \sq*{\mu(x) \phi_k(x)}_\X - \int_\X \diff{\mu(x)}{x_k} \phi_k(x) dx} \\
    &= \int_\X  \mu(x) \sum_{k=1}^d \diff{\log \pi(x)}{x_k}\phi_k(x) -
    \mu(x) \sum_{k=1}^d \diff{\log \mu(x)}{x_k} \phi_k(x) dx \;\; \text{(log trick)} \\
    &= \int_\X  \mu(x) \sq*{ \sum_{k=1}^d \phi_k(x) \pare*{ \diff{\log \pi(x)}{x_k} - \diff{\log \mu(x)}{x_k}}} dx \\
    &= \int_\X  \mu(x) \sq*{ \sum_{k=1}^d \phi_k(x) \pare*{ \diff{\log \frac{\pi(x)}{\mu(x)}}{x_k}}} dx.
  \end{split}
\end{equation}
As expected, the scale of $\E_{x \sim \mu}[\A_\pi \phi (x)]$ increases with the distance between $\mu$ and $\pi$.\\

Therefore, one can define an objective to find a density $\mu^*$ close to $\pi$:
\begin{equation}
  \mu^* = \arg\min_{\mu} \: \Sr(\mu, \pi) =
    \arg\min_{\mu} \: \max_{\phi \in \Hr} \bra*{ \E_{x \sim \mu}[\A_\pi \phi (x)] },
  \label{eq:stein_obj}
\end{equation}
as $\Sr(\mu, \pi) = 0$ iff $\mu = \pi$ and $\Sr(\mu, \pi) > 0$ otherwise with $\Hr$ sufficiently large.
The choice of $\Hr$ is therefore crucial. One way to ensure it is both rich enough and computationally tractable
is to let $\Hr$ be a RKHS.

\subsection{Kernelized Stein Discrepancy}
Let $\Hr_0$ be a RKHS with a kernel $k(x, x')$ in the Stein class of $\pi$ and $\mu$.
Let $\Hr = (\Hr^{(1)}_0, \dots, \Hr^{(d)}_0)$. The KSD maximizes $\phi$ in the unit ball of $\Hr$.
The objective in (\ref{eq:stein_obj}) is then:
\begin{equation}
  \Sr(\mu, \pi) =
    \max_{\phi \in \Hr} \bra*{ \E_{x \sim \mu}[\A_\pi \phi (x)], \; s.t. \; \|\phi\|_{\Hr} \leq 1 }.
  \label{eq:stein_ksd_obj}
\end{equation}
Within this framework, one can show that the optimal solution of (\ref{eq:stein_ksd_obj})
(see \citep{https://doi.org/10.48550/arxiv.1602.03253,
      https://doi.org/10.48550/arxiv.1410.2392,
      https://doi.org/10.48550/arxiv.1602.02964}) is:
\begin{equation}
  \phi(x) = \frac{\phi^*(x)}{\|\phi^*\|_{\Hr}}
    \text{ , where } \phi^*(.) = \E_{x \sim \mu}\sq*{ \A_\pi \otimes k(x, \cdot) }
                               = \int_\X k(x, \cdot) \nabla \log \pi(x) + \nabla k(x, \cdot) d\mu(x),
  \label{eq:stein_ksd_sol}        
\end{equation}
where $\A_\pi \otimes f(x) = f(x) \nabla \log \pi(x) + \nabla f(x)$, is a variant of Stein operator
\footnote{J'ai fait la preuve sur papier, je l'écrirai plus tard.}.
Moreover, $\Sr(\mu, \pi) = \|\phi^*\|_\Hr$.

\section{Link with Kullback-Leibler Divergence}
Let $T: \X \to \X$, $x \mapsto (I + \gamma \phi)(x)$. One can show that
(see \cite{https://doi.org/10.48550/arxiv.1608.04471} Theorem 3.1):
\begin{equation}
  \nabla_\gamma KL(T_\#\mu || \pi) = -\E_{x \sim \mu}[\A_\pi \phi(x)].
  \label{eq:grad_kl}
\end{equation}
Therefore, using (\ref{eq:stein_ksd_sol}), we know that:
\begin{equation}
  \phi^*(.) = \int_\X k(x, \cdot) \nabla \log \pi(x) + \nabla k(x, \cdot) d\mu(x)
\end{equation}
minimizes the $\nabla_\gamma KL(T_\#\mu || \pi)$.
Furthermore, assuming RKHSes $\Hr$ and $\Hr_0$ with kernel $k(x, x')$ in the Stein class of $\pi$ and $\mu$,
one can show that:
\begin{equation}
  \begin{split}
    P_\mu \nabla \log \frac{\mu}{\pi} (\cdot) &=
      \int_\X k(x, \cdot) \nabla \log \mu(x) d\mu(x) - \int_\X k(x, \cdot) \nabla \log \pi (x) d\mu(x) \\
    &= \int_\X k(x, \cdot) \nabla \mu(x) dx - \int_\X k(x, \cdot) \nabla \log \pi (x) d\mu(x) \\
    &= - \int_\X \nabla k(x, \cdot) d\mu(x) - \int_\X k(x, \cdot) \nabla \log \pi (x) d\mu(x) \\
    &= - \int_\X k(x, \cdot) \nabla \log \pi (x) + \nabla k(x, \cdot) d\mu(x) \\
    &= -\phi^*(\cdot)
  \end{split}
\end{equation}
The Stein Variational Gradient Descent (SVGD) algorithm consists
in an iterative procedure where one apply successive transformations
to an initial density $\mu_0$ towards
the "direction" $\phi^*$ that minimizes the gradient of the Kullback-Leibler divergence:
\begin{equation}
  \mu_{n+1} = \pare*{ I + \gamma \phi^* }_\# \mu_n = \pare*{ I - \gamma P_\mu \nabla \log \frac{\mu}{\pi} }_\# \mu_n.
\end{equation}

\section{Not understood yet}
\begin{itemize}
  \item $k(x, \cdot)$ in the Stein class of $\pi$ and $\mu$?
  \item Link with Wasserstein distance?
  \item Why did they defined so much about their RKHS?
\end{itemize}

\bibliography{refs}
\bibliographystyle{plainnat}



\end{document} 
